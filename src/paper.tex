\documentclass[]{cls/tools}

\usepackage[french]{babel}
\usepackage{url}

\begin{document}

\title{Un train inarrêtable}

\author{VERTUEUX}
\date{Novembre 29, 2022}

\maketitle

\section*{Introduction}

Cet article est une allégorie permettant de décrire le fondement de la société, ainsi que sa poursuite et sa 
dangerosité face au monde extérieur. Son expansion est semblable à un train progressant exponentiellement vers le bout du tunnel.

\section*{L'Allégorie}

Inventons un conte. Imaginons un homme. Il s'agit d'un homme rempli d'ambition, d'orgueil, rigoureux et 
souhaitant faire changer, progresser sa situation. Il travaillait et vivait dans une gare déprimante. 

Ces journées furent longues et ennuyeuses, particulièrement quand il conjecturait que l'administration de la gare le 
manipulait et profitait de lui, pour une raison inconnue.

Cependant, un jour, il vit un train neuf et propre. Il était équipé d'un panneau sur la portière où était marqué de ne surtout pas 
d'embarquer et de démarrer un trajet, car cela serait susceptible de mener la gare à sa perte, signé par l'administration de la gare. Telle une pomme 
proposée par un serpent. Néanmoins, d'après certains, ce train pourrait les mener vers une meilleure gare, sauf que 
l'administration refuse de leur concéder la vérité, car quitter cette gare leur donnerait l'opportunité de pouvoir découvrir 
la condition ignoble auxquels ils étaient contraints et ils pourraient d'évoluer, les mettant à un pied égal voire supérieur que ces derniers.   

L'homme, malmené et soumis à sa maléfique curiosité, décida de ne pas respecter le panneau 
et embarqua des passagers avec lui, en leur promettant notamment une gare meilleure, afin de quitter 
cette prison dépressive, pour en trouver une qui sera plus juste et valorisante. Toutefois, après 
avoir démarré le train, ce pauvre homme malencontreusement détruisit le frein à main, et le train, fragile, 
commença à avancer avec une vitesse décuplant toutes les secondes. Ce dernier devint inarrêtable. 

Les passagers, à l'arrière du train, n'eurent aucune conscience de ce qu'il se passait à l'avant, car,
ce train fermé et ne laissant aucune vue de l'extérieur, incitait naturellement les passagers à songer à d'autres
problèmes, tels que son entretien. Beaucoup veillaient pour cette gare heureuse et décente.

Excepté que certains commençaient à sentir qu'il y avait un problème, possiblement dû au ressenti de la force centrifuge
s'appliquer sur leur corps, ou un sentiment d'emprisonnement. Voici que démarre des émeutes.

Pourtant, personne ne souhaitait les écouter, ils les prenaient pour des fous pessimistes. De plus, la porte avant du train 
étant fermée, le conducteur était incapable de les informer de la situation. 

Certaines personnes conscientes auront réussi à s'enfuir du train, sauf qu'avec la grande vitesse exponentielle, cela causera bien évidemment leur perte.

La voie ferrée aura bien une fin. Cependant, pas avec la gare qu'ils attendaient éperdument.

\end{document}